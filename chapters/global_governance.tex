% !TeX spellcheck = en_GB
\documentclass[../summary.tex]{subfiles}

\begin{document}
	
	\section{Global governance}
	
	\subsection{Study guide}
	
	\begin{itemize}
		\item You should clearly understand the concepts discussed in the module and be able to recognize examples.
		\item Important concepts: principle of sustainable development, intergenerational equity, principle of sustainable use, intragenerational equity, integration principle, national
		sovereignty, principle of preventive action, precautionary principle, polluter pays principle, common but differentiated responsibility, risk assessment, environmental justice (with all its elements), collective action problem, multilateral negotiations, polycentric governance
	\end{itemize}
	
	You don’t need to read the court rulings on Bayer and chlorothalonil in chapter 3.3. The text 'Legal aspects of the choice of environmental policy instruments from the point of view of Belgian, European and international law' in chapter 4.1 does not have to be studied. You don’t need to memorize the conventions’ names or years in chapter 5.2.
	
	\subsection{Jurisdiction}
		\subsubsection{Legal principles of sustainable development}
			This part of the course introduces the principles of sustainable development based on the 1992 Rio Declaration and its updates.
			
			\paragraph{The principle of sustainable development}\mbox{}\\
				Principle three of the Rio Declaration states:
				\begin{quote}
					The right to development must be fulfilled so as to equitably meet developmental and environmental needs of present and future generations. 
				\end{quote}
				This principle originated from the World commission on Environment and Development, better known as the Brundtland Commission. It came as a response to address the accelerating deterioration of the environment. Because of the work done by this commission we have a definition of `sustainable development'. It takes the form of a three-tier concept encompassing ecological, social and economic development.\\
				\\
				In international law, sustainable development is mainly broken down into four parts:
				\begin{itemize}
					\item The principle of intergenerational equity, which amounts to the need to preserve resources for future generations. 
					\item The principle of sustainable use refers to a more immediate concern to use resources wisely.
					\item Intra-generational equity implies the balanced use of the world's resources by the various parts of the world
					\item The principle of integration implies that environmental considerations are taken into account in economic and development objectives. 
				\end{itemize}
			\paragraph{National sovereignty over natural resources}\mbox{}\\
				\label{par:national-sovereignty-over-natural-resources}
				Principle two of the Rio Declaration reads as follows:
				\begin{quote}
					States have, in accordance with the Charter of the United Nations and the principles of international law, the sovereign right to exploit their own resources pursuant to their own environmental and developmental policies, and the responsibility to ensure that activities within their jurisdiction or control do not cause damage to the environment of other States or of areas beyond the limits of national jurisdiction
				\end{quote}
				This essentially means that the international laws about sustainability have State sovereignty over its own resources as a cornerstone. However, this does not mean that States can do whatever they like, they can not inflict damage to the territory of other States for example. Environmental challenge the classic theory of international law regarding territory and State sovereignty.  This is why there is a relatively new part of a States sovereignty: the protection of the global commons. This was added to the traditional three regimes vis-à-vis jurisdiction:
				\begin{itemize}
					\item The majority of the Earth is subject to territorial sovereignty.
					\item Res nullius are those parts of the Earth which are capable of lawful national appropriation/sovereignty, but are as yet unclaimed.
					\item Res communis are shared by all nations and cannot be placed under State sovereignty
				\end{itemize}
				Especially the distinction between global commons and res communis is relevant. The largest challenge is to find a way for parties to adjudicate regulatory power over the global commons. There are some concerns about:
				\begin{itemize}
					\item Alleviates veto concerns: one can not assume that global consensus will be found over issues concerning global commons. If this was the case, every State would effectively hold veto power over the management of these. 
					\item Unbridled unilateralism or even selective multilateralism indeed would risk rewarding coercion.
				\end{itemize}
				
			\paragraph{The principle of preventive action and the precautionary principle}\mbox{}\\
				Principle two seen in paragraph \ref{par:national-sovereignty-over-natural-resources} in combination with principle 15 of the Rio Declaration define the prevention principle: 
				\begin{quote}
					In order to protect the environment, the precautionary approach shall be widely applied by States according to their capabilities. Where there are threats of serious or irreversible damage, lack of full scientific certainty shall not be used as a reason for postponing cost-effective measures to prevent environmental degradation
				\end{quote}
				This principle obliges authorities to take action at the earliest possible stage to prevent known risks from being realized. There however is not an undisputed definition of the precautionary principle. Generally though, it is defined in a more negative sense: States must not defer regulatory action even if there is no conclusive scientific proof between a given (in)action and damage to the human health or environment.\\
				\\
				As for the content of the principles, they can be distinguished in terms of the types of risks one has to manage. Preventative deals with known risk and precautionary deals with uncertain risks. The former is also part of international law while the latter is again disputed. 
	\subsection{Law and science}
	
	\subsection{Regulation vs. liability}
	
	\subsection{Environmental justice}
	
	\subsection{Multilateral negotiations}
	
	\subsection{The bigger picture}

\end{document}