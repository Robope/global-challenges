\documentclass[../summary.tex]{subfiles}

\begin{document}
	
	\section{Global governance}
	
	\subsection{Study guide}
	
	\begin{itemize}
		\item You should clearly understand the concepts discussed in the module and be able to recognize examples.
		\item Important concepts: principle of sustainable development, intergenerational equity, principle of sustainable use, intragenerational equity, integration principle, national
		sovereignty, principle of preventive action, precautionary principle, polluter pays principle, common but differentiated responsibility, risk assessment, environmental justice (with all its elements), collective action problem, multilateral negotiations, polycentric governance
	\end{itemize}
	
	You don’t need to read the court rulings on Bayer and chlorothalonil in chapter 3.3. The text 'Legal aspects of the choice of environmental policy instruments from the point of view of Belgian, European and international law' in chapter 4.1 does not have to be studied. You don’t need to memorize the conventions’ names or years in chapter 5.2.
	
\end{document}