% !TeX spellcheck = en_GB
\documentclass[../summary.tex]{subfiles}

\begin{document}
	
	\section{Global governance}
	
	\subsection{Study guide}
	
	\begin{itemize}
		\item You should clearly understand the concepts discussed in the module and be able to recognize examples.
		\item Important concepts: principle of sustainable development, intergenerational equity, principle of sustainable use, intragenerational equity, integration principle, national
		sovereignty, principle of preventive action, precautionary principle, polluter pays principle, common but differentiated responsibility, risk assessment, environmental justice (with all its elements), collective action problem, multilateral negotiations, polycentric governance
	\end{itemize}
	
	You don’t need to read the court rulings on Bayer and chlorothalonil in chapter 3.3. The text 'Legal aspects of the choice of environmental policy instruments from the point of view of Belgian, European and international law' in chapter 4.1 does not have to be studied. You don’t need to memorize the conventions’ names or years in chapter 5.2.
	
	\subsection{Jurisdiction}
		\subsubsection{Legal principles of sustainable development}
			This part of the course introduces the principles of sustainable development based on the 1992 Rio Declaration and its updates.
			
			\paragraph{The principle of sustainable development}\mbox{}\\
				Principle three of the Rio Declaration states:
				\begin{quote}
					The right to development must be fulfilled so as to equitably meet developmental and environmental needs of present and future generations. 
				\end{quote}
				This principle originated from the World commission on Environment and Development, better known as the Brundtland Commission. It came as a response to address the accelerating deterioration of the environment. Because of the work done by this commission we have a definition of `sustainable development'. It takes the form of a three-tier concept encompassing ecological, social and economic development.\\
				\\
				In international law, sustainable development is mainly broken down into four parts:
				\begin{itemize}
					\item The principle of intergenerational equity, which amounts to the need to preserve resources for future generations. 
					\item The principle of sustainable use refers to a more immediate concern to use resources wisely.
					\item Intra-generational equity implies the balanced use of the world's resources by the various parts of the world
					\item The principle of integration implies that environmental considerations are taken into account in economic and development objectives. 
				\end{itemize}
	
	\subsection{Law and science}
	
	\subsection{Regulation vs. liability}
	
	\subsection{Environmental justice}
	
	\subsection{Multilateral negotiations}
	
	\subsection{The bigger picture}

\end{document}