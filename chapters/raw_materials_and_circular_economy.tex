\documentclass[../summary.tex]{subfiles}

\begin{document}
	
	\section{Raw materials and circular economy}
	
	\subsection{Study guide}
	
	Make sure to understand:
	\begin{itemize}
		\item  Meaning of the terms (non-)renewable resources, (in-)finite resources, critical materials
		\item The drivers of material demand (i.e. understand the I-PAT equation and past and predicted evolutions of the constituent factors), including also influence of the energy transition on predicted material demand
		\item Different types of scarcity, including examples
		\item Potential caveats of using biobased materials
		\item Order of magnitude of waste produced and of importance of waste treatment strategies
		\item Waste hierarchy
		\item Scope of the Basel convention
		\item The importance of resource demand in terms of environmental impact (climate change, water stress, biodiversity).
		\item Strategies to reduce climate impact of steel and cement production (= materials with highest cumulative contribution to climate change)
		\item Life Cycle Analysis: what is the meaning of the 4 main phases (goal and scope, inventory, impact assessment, interpretation), the concept of “functional unit”; and what is the difference between
		an emission, an impact category (= mid-point) and a damage category (= endpoint)
		\item Difference between relative decoupling and absolute decoupling
		\item Circular economy strategies
		\item Reasons why recycling is not going to solve all material resource related problems
		\item Difference between Direct Material Consumption and Material Footprint
		\item Links with other challenges
	\end{itemize}
	
	\subsection{(Non-)renewable resources, (in-)finite resources and critical materials}
	
	
	
	\subsection{Drivers of material demand}
	
	There are three factors that drive the evolution in material demand: \textbf{population}, \textbf{affluence} and \textbf{technology}. 
	\\\\
	Population represents the number of people on the globe. The population will continue to keep growing, and paired with this grow is a higher demand for materials. However, the annual growth rate of the world population is decreasing so the importance of population growth in the evolution in our material demand will also decrease in the future.
	\\\\
	Affluence represents the average level of wealth per person, often expressed as the GDP in dollar per capita. The worldwide growth rate of GDP per person has remained rather constant over time. Because of this, we can extrapolate these numbers for the foreseeable future.
	\\\\
	The last factor represents the contribution of technology, specifically how material-intensive our economy is and how many kg of material do we need per dollar of GDP. We can see that industrialized countries have a higher material footprint and a higher GDP than average. If we look at this relatively however, though their total material footprint still increases, their relative material footprint per dollar of GDP decreases. This is not the case in rural societies. The reason behind this is that maintaining something is less material intensive that developing something and there are innovate inventions by scientists and engineers, who develop new technologies that require less material. 
	\\\\
	If we look at the reasons behind material demand, we can conclude that the demand will continue to keep growing. This growth will not be equal for all material types. Consider for instance materials needed for the energy transition. Scientists are expecting significantly higher growth rates because of the transition to a carbon neutral energy supply.
	
	\subsection{Different types of scarcity}
	
	Absolute scarcity is when the amount of raw materials within these reservoirs can only cover the demand for that material for the coming decade or two. There can be a couple reasons for this. First and foremost, the most easily accessible and richest raw materials do deplete. The quality of raw materials is often decreasing. Copper ore, for example, contained 3\% copper at the beginning of the last century; today it contains a good 0.3\%. So, ten times as much copper ore has to be mined for the same amount of copper. 
	\\\\
	A second aspect of scarcity is that some ores are concentrated in only a limited number of countries. Regional conflicts can have a major impact on the global price of the raw material. We say such materials are economically or geopolitically scarce. For example, China used an embargo on rare earth exports as an economic threat during the dispute with Japan over the Senkaku Islands in 2010. As a result, the price of some elements increased a tenfold in a matter of months.
	\\\\
	Finally, a third scarcity problem is that quite some more exotic elements, often used in high-tech applications, cannot be mined by themselves. They occur in small quantities as a by-product of ores of other metals, called carrier metals. We therefore say that indium is structurally scarce. An example of a structurally scarce element is indium, which is a by-product of zinc.
	
	\subsection{Potential caveats of using biobased materials}
	
	Tthere are a number of caveats related to the use of biobased resources. Cultivated or growing \textbf{biomass needs land area}. The land used to grow maize, for example, from which PLA plastic can be made, cannot be used for other crops. A competition may arise between the use of land for materials or for food. Furthermore, soil can become exhausted if used too intensively, and there also will be a impact on biodiversity if new land is cultivated for new crops for materials. 
	\\\\
	Also typical of agriculture is \textbf{the use of fertilisers and pesticides}, which can make the environmental impact of bio-based materials significant.
	\\\\
	Other concerns are about the \textbf{water and energy needed}. Some crops for raw materials, for example cotton, require a lot of water during cultivation. Most crops also require quite a bit of additional energy for tilling the soil, harvesting, and so on. Some new types of bio-based plastics even require more energy and more CO2 emissions to make them than an equivalent conventional petroleum-based plastic. 
	\\\\
	We can conclude that replacing non-renewable materials by bio-based materials is promising, but the impact throughout the entire life cycle has to be carefully investigated.
	
	\subsection{Waste}
	
	The world generates more than 2 billion tonnes of municipal solid waste annually. In 2020, an amount of 2.24 billion was estimated. The amount of waste production is different in each country and is mainly related to their income levels. Municipal solid waste includes household, commercial and institutional waste. The two biggest waste streams when it comes to solid waste include electronic waste and plastic waste.
	\\\\
	Electronic waste is one of the fastest-growing waste streams globally. However, it is not always managed correctly, and often is exported to developing countries where it is managed in uncontrolled and unsafe ways. To address this issue, \textbf{the Basel Convention} was established to ensure sound environmental management of electronic waste and to prevent its illegal export to those countries.
	\\\\
	The current trends in waste generation and management have significant environmental, social, and economic impacts. If you think of the more than 2 billion tons of waste produced, over 30\% of this waste is still not managed in an environmentally safe manner. This has a lot of negative impacts like a loss of resources, as well as to different environmental and health impacts due to soil and water contamination, and air pollution. For example, solid waste is responsible for 3\% of global greenhouse gas emission. Furthermore, the degradation of organic waste in landfills (and open dumps) generates a gas composed mainly of carbon dioxide and methane. Another example is that waste releases hazardous emissions and particular matter into the air when it is burned in an uncontrolled manner, causing significant health issues. Lastly, unmanaged waste degradation can result in the creation and runoff of leachate, a hazardous liquid that can contaminate water bodies and soil, leading to the contamination of, for example, drinking water, and transmission of diseases.
	\\\\
	In Europe the European Commission defined the \textbf{waste hierarchy} as a guiding principle for sustainable and integrated waste management. It ranks waste management alternatives based on their potential to minimize environmental impacts, health risks, and promote the recovery of resources. \\
	At the top of the hierarchy, waste prevention should be prioritized to minimize waste generation. If prevention is not possible, the focus should shift towards promoting the reuse of materials and components. Recycling follows as the next best solution. When none of the above options are feasible, energy recovery from waste is considered. Finally, landfilling represents the least preferable alternative, and waste disposal should be minimized as much as possible.
	\\\\
	To conclude, solid waste management is crucial to reduce the negative effects on the environment and human health, and support the transition to a clean and circular economy by promoting resource recovery.  
	
	\subsection{Importance of resource demand}
	
	
	
	\subsection{Reduction of steel and cement productions climate impact}
	
	Cement and steel play a large part in the greenhouse gas emission, and it’s use is still increasing. What are possible ways to reduce these emissions?
	
	\begin{enumerate}
		\item A part of the emissions come from the energy used during production: Decrease production energy by more efficient processes, and use more renewable energy sources for the energy needed during this production processes. 
		\item However, there are also emissions related to the production of the material itself. For example, during the production of cement, a chemical process takes place during which CO2 is released. During steel production, cokes are used in the chemical process to extract iron from iron ore, thereby releasing large amounts of CO2. A possible solution is using different resources during these processes. For steel production this is already possible by using hydrogen instead of cokes. However, this requires large investments in new production processes and infrastructure and the production of hydrogen also requires some energy. Another way to reduce the emission of CO2 is by capturing it before it is released into the atmosphere and storing it (Carbon Capture Storage – CCS) or using it elsewhere (Carbon Capture and Utilization – CCU). 
		\item A third solution lies in reusing and recycling of waste, so that the demand for new materials decreases. 
	
	\end{enumerate}
	
	\subsection{Life Cycle Analysis}
	
	Life Cycle Analysis is a tool that allows us to systematically compare alternative design concepts and system designs, based on scientific considerations, when our concern is impact and especially impact avoidance.
	
	There are main phases in this procedure. The first one is \textbf{goal and scope definition}, where you need to very clearly determine what is in and what is outside your study. Secondly, once the boundaries are clear, you go for \textbf{inventory}. This can be fairly tedious, sometimes it takes months or even years to collect all the relevant data, but it’s indispensable. We have to get through that data collection stage and only then we can come to the actual \textbf{impact analysis} where we ask ourselves how much these different ins and outs contribute to the different impact categories that we discussed before. The last phase is the \textbf{interpretation}.
	
	\subsection{Relative and absolute decoupling}
	
	The concept of decoupling is a strategy put forward by the United Nations, and by the European Commission. The goal is that the need for resources doesn’t follow the growth of economy. Decoupling is not only put forward for resource use, but also for the environmental impact. The UN calls it a dual decoupling: the decoupling of resource use from well-being on the one hand, and impact decoupling decreasing environmental and social impacts per unit of resource use on the other hand.
	\\\\
	When the need for resources still rising, but at a lower rate than the economy, we have a \textbf{relative decoupling} of resource needs from the economy.
	\\\\
	When the need for resources decreases in absolute terms, while the economy is growing, than we have \textbf{absolute decoupling}.
	\\\\
	If we look at reality, we can see that resource use is decoupled from economic growth in Europe. On a global scale however, there is no decoupling at all between resource use and economy, rather on the contrary. This doesn't mean that Europe is doing better than the rest of the world, because the reason for this is that a lot of Europe's resources are imported from other continents and hence excluded from it's \textbf{direct material consumption}. Another way of measuring our materials use is by the \textbf{material footprint}, in which all materials consumed in the supply chain to make products, wherever that happened, is included. Typically, the material footprint for high income countries is higher than the direct material consumption, and the decoupling of material footprint from GDP or economic growth is less obvious.
	
	\subsection{Circular economy}
	
	A commonly accepted definition of circular economy says that it is an economy in which the value of materials in the economy is maximised and preserved for as long as possible. This means that the input of new materials and their consumption are minimised, that waste generation is minimised and negative and that environmental impacts are reduced throughout the life cycle of materials. 
	\\\\
	This can be done by decreasing the number of needed products by keeping products longer in use, by increasing the lifetime by easier repairing during its life and by reuse. Or products could be used more intensively, by sharing its idle time with other people or by only borrowing the product when you really need it.
	\\\\
	Another way is making the same or similar products with a smaller amount of materials. Strategies to do this are ecodesign of products and process intensification in order to reduce the waste during production. Materials are also saved when components from a discarded product still are recovered and remanufactured.
	\\\\
	Finally, materials and alloys are made out of raw materials. A last question then is whether we can reduce the input of new raw materials, and replace them by more sustainable, renewable or recycled raw materials. 
	\\\\
	Recycling is a cornerstone, but however, the last option of the circular economy strategies. In the circular economy it is even better not to have to discard material, or at least postpone that moment of disposal as much as possible. Preserving value is the main goal.
	
\end{document}