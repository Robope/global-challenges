\documentclass[../summary.tex]{subfiles}

\begin{document}
	
	\section{Raw materials and circular economy}
	
	\subsection{Study guide}
	
	Make sure to understand:
	\begin{itemize}
		\item  Meaning of the terms (non-)renewable resources, (in-)finite resources, critical materials
		\item The drivers of material demand (i.e. understand the I-PAT equation and past and predicted evolutions of the constituent factors), including also influence of the energy transition on predicted material demand
		\item Different types of scarcity, including examples
		\item Potential caveats of using biobased materials
		\item Order of magnitude of waste produced and of importance of waste treatment strategies
		\item Waste hierarchy
		\item Scope of the Basel convention
		\item The importance of resource demand in terms of environmental impact (climate change, water stress, biodiversity).
		\item Strategies to reduce climate impact of steel and cement production (= materials with highest cumulative contribution to climate change)
		\item Life Cycle Analysis: what is the meaning of the 4 main phases (goal and scope, inventory, impact assessment, interpretation), the concept of “functional unit”; and what is the difference between
		an emission, an impact category (= mid-point) and a damage category (= endpoint)
		\item Difference between relative decoupling and absolute decoupling
		\item Circular economy strategies
		\item Reasons why recycling is not going to solve all material resource related problems
		\item Difference between Direct Material Consumption and Material Footprint
		\item Links with other challenges
	\end{itemize}
	
	\subsection{(Non-)renewable resources, (in-)finite resources and critical materials}
	
	\subsection{Drivers of material demand}
	There are three factors that drive the evolution in material demand: \textbf{population}, \textbf{affluence} and \textbf{technology}. 
	\\
	\\
	Population represents the number of people on the globe. The population will continue to keep growing, and paired with this grow is a higher demand for materials. However, the annual growth rate of the world population is decreasing so the importance of population growth in the evolution in our material demand will also decrease in the future.
	\\
	\\
	Affluence represents the average level of wealth per person, often expressed as the GDP in dollar per capita. The worldwide growth rate of GDP per person has remained rather constant over time. Because of this, we can extrapolate these numbers for the foreseeable future.
	\\
	\\
	
	
	\subsection{Different types of scarcity}
	
	\subsection{Potential caveats of using biobased materials}
	
	\subsection{Waste}
	
	\subsection{The Basel convention}
	
	\subsection{Importance of recourse demand}
	
	\subsection{Reduction of steel and cement productions climate impact}
	
	\subsection{Life Cycle Analysis}
	
	\subsection{Relative and absolute decoupling}
	
	\subsection{Circular economy}
	
	\subsection{Recycling}
	
	\subsection{Direct Material Consumption versus Material Footprint}
	
	\subsection{Links with other chapters}
	
	
\end{document}