\documentclass[../summary.tex]{subfiles}

\begin{document}
	
	\section{Raw materials and circular economy}
	
	\subsection{Study guide}
	
	Make sure to understand:
	\begin{itemize}
		\item  Meaning of the terms (non-)renewable resources, (in-)finite resources, critical materials
		\item The drivers of material demand (i.e. understand the I-PAT equation and past and predicted evolutions of the constituent factors), including also influence of the energy transition on predicted material demand
		\item Different types of scarcity, including examples
		\item Potential caveats of using biobased materials
		\item Order of magnitude of waste produced and of importance of waste treatment strategies
		\item Waste hierarchy
		\item Scope of the Basel convention
		\item The importance of resource demand in terms of environmental impact (climate change, water stress, biodiversity).
		\item Strategies to reduce climate impact of steel and cement production (= materials with highest cumulative contribution to climate change)
		\item Life Cycle Analysis: what is the meaning of the 4 main phases (goal and scope, inventory, impact assessment, interpretation), the concept of “functional unit”; and what is the difference between
		an emission, an impact category (= mid-point) and a damage category (= endpoint)
		\item Difference between relative decoupling and absolute decoupling
		\item Circular economy strategies
		\item Reasons why recycling is not going to solve all material resource related problems
		\item Difference between Direct Material Consumption and Material Footprint
		\item Links with other challenges
	\end{itemize}
	
	\subsection{(Non-)renewable resources, (in-)finite resources and critical materials}
	
	
\end{document}