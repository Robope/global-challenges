\documentclass[../summary.tex]{subfiles}

\begin{document}
	
	\section{Demography}
	
	\subsection{Study guide}
	
	Module 3 (Demography) sketches basic insights into the origins and dynamics of population growth. Make sure to understand:
	\begin{itemize}
		\item The link between the industrial revolution and evolutions in mortality
		\item The drivers behind the evolution of fertility rates
		\item The link between evolutions of mortality and fertility
		\item How evolutions of mortality and fertility influence both the demographics and the economic
		growth expectations of societies
	\end{itemize}
	Don’t learn the specific examples by heart; focus on understanding the reasons behind the evolutions.
	
	\subsection{Link between industrial revolution and evolution in mortality}
	The spread of the industrial revolution was paired with a decline in mortality.  
	
	\subsection{Drivers behind the evolution of fertility rates}
	The first driver behind the evolution of fertility rates is a \textbf{reduction in child mortality}. The reason for this is that, when you grow older but there are no pensions or health care, you need your children to support you during old age. When there is a high child mortality, you will therefore need to make more children if you want a certain amount to survive.  
	\\
	\\
	Another driver behind the evolution of fertility rates is the \textbf{wealth of a country}. Wealthier societies have less children. 
	\\
	\\
	We also see that there is a relationship with \textbf{marriage age} in societies where women marry later, they have less children. That's not unexpected either. But we also see a very strong relationship with the \textbf{education that is offered to women}. If women are educated, they are better capable of applying of using contraception correctly. If you use contraception correctly, then you will end up probably with less children. It is also true that if you offer education to women, then of course they need to invest time in that and very few people have children while they study. Also, after their studies, these women very often want to have a job and pursue a career. Because of that, they will have less time to care for children and if there is less time available, that will also lead to having less children. We can see that there are a lot of reasons why giving a quality eduction to woman leads to a decline in fertility rates.
	
	\subsection{Link between evolutions of mortality and fertility}
	In general the decline of fertility starts decades after the decline of mortality. Society has to adjust to a higher life expectation before this higher life expectation translates into a lower number of children being born. The only exception to this rule is France in the end of the 18th century where mortality and fertility started to decline at the same time. Scientist suggest that there is a link with the French Revolution. 
	
	\subsection{Influence of mortality and fertility evolution on the demographic and economic growth expectations of societies}
	
\end{document}