% !TeX spellcheck = en_GB
\documentclass[../summary.tex]{subfiles}

\begin{document}
	\section{Human behaviour}
	
		\subsection{Study guide}
			A lot of the discussed global challenges and their solutions require a lot of change in behaviour from us humans. This however, is not easy. Because of this it's important to understand how the human brain works, what drives us and what motivates us. \\
			\\
			We will first explore the way in which we make decisions when faced with different dilemma's or situations. Once we know more about this, we can take a look at how to change these thoughts and convince people. Finally, we will take a look at how biasses affect our thinking. 
		\subsection{Conditioning vs. motivating}
			\textbf{Edward Thorndyke} was one of the most important psychology researchers to live. He started his career at Harvard studying ``the instinctive and intelligent behaviour of chickens''. Moving on to the study of cats at Columbia University. In his experiments with cats, he deprived them of food and let them solve some sort of puzzle to get access to food. When he would repeat this sequence with the same cats, he noticed they would very quickly catch on to the solution, and by the third repetition, they could solve the puzzle in a matter of seconds. \\
			\\
			Similar and systematic experiments gave way to the \textbf{law of effect}: `responses that produce a satisfying effect in a particular situation become more likely to occur again in that situation, and responses that produce a discomforting effect become less likely to occur again in that situation'.\\
			\\
			Later on this law was studied by \textbf{Boris Frederick Skinner}.   He discovered the animals which were most likely to have good learning results with rewards: rats, mice, pigeons and pigs. He also pioneered a way to prove learned behaviour by systematically withholding reward. \\
			\\
			The act of learning and unlearning was central in psychology research for many decades. Thousands of papers were written about the effectiveness of rewarding because it was a good motivator in the tested animal groups. 
			\subsubsection{Motivation}
				When it comes to \textbf{rewarding behaviour}, a complication is that people overestimate the importance of \textbf{extrinsic motives} in the behaviour of others, while they look at their own behaviour as more \textbf{intinsically motivated}.\\
				\\
				A study by Chip Heath illustrated this by asking people to rank the order of importance of eight motives in their professional life. The first four were extrinsic: pay, benefits, praise and security; the four others were intrinsic: learning, developing, skills and feeling good with worthwhile work. On average, the first four out of five were intrinsic motivators, pay was only ranked on the fourth place. \\
				\\
				When the participants were asked to do this for their peers, pay moved up to the second place. Going even further, when they were asked to do this for their managers or bank clerks, they ranked pay on place one. Those same managers and bank clerks only ranked pay on position seven for themselves. \\
				\\
				Because of these insights, most psychologists think of rewards and sometimes punishment to aid behavioural change in people. But getting people intrinsically motivated for change is more qualitative and at least as important as the reward structure. 
		\subsection{Social dilemmas}
		
		\subsection{Individual differences}
		
		\subsection{Thoughtful or automatic}
		
		\subsection{Persuasive communication}
		
		\subsection{Stakeholder system}
		
		\subsection{Biases}
		
		\subsection{Questions and feedback}
\end{document}