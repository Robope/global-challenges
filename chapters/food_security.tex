\documentclass[../summary.tex]{subfiles}

\begin{document}
	
	\section{Food security}
	
	\subsection{Study guide}
	
	Students should understand important concepts such as:
	\begin{itemize} 
	\item Food security
	\item Food gap
	\item Land gap
	\item Yield gap
	\item Synthetic versus bio-based fertilizers
	\item Intensive versus extensive farming
	\item Classic breeding technologies versus genetically modified crops
	\item Plant protection products (PPP)
	\item Integrated pest management (IPM)
	\item Sustainable diet
	\item Protein shift
	\item Food waste
	\item Generational renewal
	\end{itemize}
	In addition, students should understand the environmental impact of food production / agriculture (land use, greenhouse gas emission, nutrient emission by synthetic fertilizers), including differences between animal-based versus plant-based foods.
	\\\\
	Finally, students should have insight into solutions to increase food security (increase yields, adopt healthy
	diets, reduce food waste)
	
	\subsection{Food security}
	
	Food security exists when all people, at all times, have physical and economic access to sufficient, safe and nutritious food that meets their dietary needs and food preferences for an active and healthy life. 
	
	\todo[inline]{Add part about food security }
	
	\subsection{Socio-economic characteristics}
	
	
	
	\subsection{Environmental impact of agriculture}
	
	\subsection{Sustainable food consumption}
	
\end{document}